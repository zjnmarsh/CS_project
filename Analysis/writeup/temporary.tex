\subsection{SQL queries used}
\textbf{Creating a new table}
\begin{lstlisting}
CREATE TABLE users (username text PRIMARY KEY, see_all integer)

CREATE TABLE ca_param (
    user string,
    no_cells integer,
    generations integer,
    size_x integer,
    size_y integer,
    infection_radius integer,
    no_infected integer,
    recovered_can_be_infected integer,
    days_until_recovered integer,
    use_immunity integer,
    days_of_immunity integer    
    )

CREATE TABLE sir_param (
    user string,
    sus0 integer,
    inf0 integer,
    rec0 integer,
    beta integer,
    gamma integer,
    time integer
    )

INSERT INTO users VALUES (:username, :see_all), {'username': 'admin', 'see_all': 1}
\end{lstlisting}
This code sets up a clean database containing 3 tables, users, ca_param and sir_param. In the users table there are two columns, one containing the username and another containing a value to decide whether that user can see the history of all calculations performed and not just the calculations performed by that user. THIS WILL BE UPDATED. The ca_param and sir_param table contain columns to record the user doing the calculation and the parameters the user entered.

\textbf{Inserting a new user into the user table}
\begin{lstlisting}
INSERT OR IGNORE INTO users VALUES (:username, :see_all),{'username': in_user, 'see_all': 0}
\end{lstlisting}

\textbf{Entering parameters from the Cellular Automata model}
\begin{lstlisting}
INSERT INTO ca_param VALUES (:user, :no_cells, :generations, :size_x, :size_y,
                                            :infection_radius, :no_infected, :recovered_can_be_infected,
                                            :days_until_recovered, :use_immunity, :days_of_immunity),
              {'user': in_user, 'no_cells': up[0], 'generations': up[1], 'size_x': up[2], 'size_y': up[3],
               'infection_radius': up[4], 'no_infected': up[5], 'recovered_can_be_infected': up[6],
               'days_until_recovered': up[7], 'use_immunity': up[8], 'days_of_immunity': up[9]}
\end{lstlisting}
This function will be passed the user, in_user, and a list of the parameters entered, up. This will be added into the ca_param table.